%
% Resume
% Contents
%

%----------HEADING-----------------
\begin{tabular*}{\textwidth}{l@{\extracolsep{\fill}}r}
  \textbf{\href{https://mrzzy.co/}{\Large 朱展言}} & 邮箱: \href{mailto:dev@mrzzy.co}{dev@mrzzy.co}\\
  \href{https://mrzzy.co/}{软件工程师(数据方向)} & GitHub: \href{https://github.com/mrzzy}{github.com/mrzzy} \\
  & Linkedin: \href{https://www.linkedin.com/in/zhu-zhanyan/}{linkedin.com/in/zhu-zhanyan} \\
\end{tabular*}


一名以数据工程为重点的计算机科学本科生,具备强大的软件工程背景,热衷于通过技术解决复杂问题,具有一定的机器学习经验和数据处理能力。

%-----------EXPERIENCE-----------------
\section{工作经历}
  \resumeSubHeadingListStart
  
    \resumeSubheading
      {Gojek}{}
      {机器学习工程师实习生}{2020年3月 - 2020年10月}
      \resumeItemListStart
        
          \item{为Gojek的司机-乘客匹配模型构建了一个开源机器学习特征库,提升了数据处理效率,使用了Python、Java、Golang、Kafka和Redis等技术。}
        
          \item{解决了一个关键的生产问题,通过修正Redis存储架构不匹配,确保了匹配模型的稳定运行。}
        
          \item{使用Grafana和GCP BigQuery设计并部署了一个综合的分析仪表板,帮助管理层获得实时数据洞察,支持数据驱动决策。}
        
      \resumeItemListEnd
  
    \resumeSubheading
      {Tinkertanker}{}
      {软件工程师实习生}{2018年1月 - 2018年4月}
      \resumeItemListStart
        
          \item{通过使用Python、Django和PostgreSQL,重构了一个传统的讲师-课程分配系统,极大地提升了系统的性能和可靠性,取代了之前基于Google Sheets的方案。}
        
          \item{使用Jinja、HTML和JavaScript开发了响应式用户界面,优化了用户体验,提升了系统的可操作性。}
        
          \item{实现了最佳开发实践,包括全面的测试、文档编写和通过Github Actions进行的持续集成,延长了项目生命周期超过5年。}
        
      \resumeItemListEnd
  
  \resumeSubHeadingListEnd


%-----------PROJECTS-----------------
\section{项目经验}
  \resumeSubHeadingListStart
    
      \resumeSubheading
        {Flowmotion}{2024年9月 - 2024年11月}
        {\href{https://mrzzy.co/work/flowmotion-real-time-routing-for-sg-drivers}{https://mrzzy.co/work/flowmotion-real-time-routing-for-sg-drivers}}
        {\href{https://github.com/ZhiXin18/flowmotion}{https://github.com/ZhiXin18/flowmotion}}
        \resumeItemListStart
          
            \item{开发了一个实时交通拥堵路由应用,利用实时摄像头数据和机器学习优化新加坡司机的行驶路线,提高通勤效率。}
          
            \item{集成了一个自定义的基于YOLOv8的机器学习模型来分析交通拥堵,结合OSRM API为用户提供实时路线优化。}
          
            \item{使用Flutter开发了跨平台移动应用,确保在iOS和Android上无缝运行,同时通过Firebase和Google Cloud管理可扩展的云基础设施。}
          
        \resumeItemListEnd
    
      \resumeSubheading
        {Providence}{2023年11月 - 2024年11月}
        {\href{https://mrzzy.co/work/providence-data-engineering-personal-finance}{https://mrzzy.co/work/providence-data-engineering-personal-finance}}
        {\href{https://github.com/mrzzy/providence}{https://github.com/mrzzy/providence}}
        \resumeItemListStart
          
            \item{设计并自动化了一个强大的ELT管道,将多个来源的个人财务数据进行整合,提供数据驱动的财务洞察,帮助用户做出更明智的预算决策。}
          
            \item{通过迁移到DuckDB优化了云架构,降低了数据存储和查询成本,同时提升了查询性能,确保了系统可扩展且运维成本低。}
          
            \item{使用Apache Superset构建了一个动态财务仪表盘,为用户提供关于消费趋势和财务健康的实时洞察,并提供了友好的界面来进行个人财务管理。}
          
        \resumeItemListEnd
    
  \resumeSubHeadingListEnd

%-----------EDUCATION-----------------
\section{教育背景}
  \resumeSubHeadingListStart
    
      \resumeSubheading
        {南洋理工大学}{}
        {计算机科学 本科}{2023年8月 - 至今}
    
      \resumeSubheading
        {义安理工学院}{}
        {信息技术 大专}{2018年4月 - 2021年4月}
    
  \resumeSubHeadingListEnd

%--------SKILLS------------
\section{技能掌握}
 \resumeSubHeadingListStart
   
     \resumeSubItem{后端技术}{Python, SQL, Java, JavaScript, TypeScript, Node.js, Golang}
    
     \resumeSubItem{云计算}{GCP, Firebase, AWS, Terraform, Docker, Ansible, Packer, Kubernetes, CI/CD, Github Actions}
    
     \resumeSubItem{数据工程}{PostgreSQL, Redis, BigQuery, Grafana, DBT, Prefect, Spark, DuckDB, Firestore, 数据建模, 数据仓库}
    
     \resumeSubItem{其他}{Linux, 云基础设施, 数据分析}
    
     \resumeSubItem{软技能}{具备出色的分析和解决问题能力,能够从复杂问题中提取关键点并提出创新的解决方案;勤奋踏实,能够在压力下持续高效工作;具有良好的团队合作精神,善于与不同背景的团队成员协作,共同推动项目的成功;良好的自学能力,能够迅速适应新技术和工具,并将其应用于实际项目中;积极乐观,具备强烈的责任感和良好的沟通能力。}
    
 \resumeSubHeadingListEnd

% vim:ft=tex.jinja2
